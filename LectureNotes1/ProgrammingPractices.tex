\documentclass[english,xcolor=dvipsnames]{beamer}
% load package with ``framed'' option.
\usepackage[framed,autolinebreaks,useliterate]{mcode}
\usepackage[orientation=landscape,size=custom,width=16,height=9,scale=0.48,debug]{beamerposter}
\usepackage[T1]{fontenc}
\usepackage[latin9]{inputenc}
\usepackage{amsthm}
\usepackage{amsmath}
\usepackage{amssymb}
\usepackage{bookmark}
\usepackage{graphics,graphicx}
\usepackage{pstricks,pst-node,pst-tree}
\usefonttheme{serif}
\usepackage{palatino}
\usepackage{tikz}
\usetikzlibrary{shapes,arrows}
\usetikzlibrary{positioning}
%\usepackage[margin=.5cm]{geometry}

\definecolor{dgreen}{rgb}{0.,0.6,0.}
\definecolor{forest}{RGB}{34.,139.,34.}
\definecolor{byublue}{RGB}{0.,30.,76.}
\definecolor{dukeblue}{RGB}{0.,0.,156.}
%\usetheme{Ilmenau}
\usetheme{Warsaw}
\usecolortheme[named=dukeblue]{structure}
%\usecolortheme[named=RawSienna]{structure}
%\usecolortheme[named=byublue]{structure}
\setbeamertemplate{navigation symbols}{}
\setbeamertemplate{footline}{}
\setbeamercovered{transparent}

\newcommand{\be}{\begin{enumerate}}
\newcommand{\ee}{\end{enumerate}}
\newcommand{\bq}{\begin{quote}}
\newcommand{\eq}{\end{quote}}
\newcommand{\bd}{\begin{description}}
\newcommand{\ed}{\end{description}}
\newcommand{\bi}{\begin{itemize}}
\newcommand{\ei}{\end{itemize}}

\title[]{Programming Practices}
\author{Tyler Ransom\thanks{Taken from Justin Trogdon}}
\date{\today}

\begin{document}

\begin{frame}
   \titlepage
\end{frame}


\begin{frame}
\frametitle{Steps to good programming}

   \bi 
   \item Plan
   \item Develop
   \item Test
   \item Time should be spent equally on all three steps
   \ei
\end{frame}

\begin{frame}
\frametitle{Plan}

   \bi 
   \item Make a road map
      \bi 
      \item What is the desired output?
      \item What variables will be needed?
      \item Are there components that will be used again?
      \ei
   \item Keep components small and manageable
   \ei
\end{frame}

\begin{frame}
\frametitle{Develop}

   \bi 
   \item Take advantage of existing routines (e.g. \mcode{regress})
   \item Be efficient
      \bi 
      \item Vectorize code in Matlab
      \ei
   \item Include detailed comments (see below)
   \ei
\end{frame}

\begin{frame}
\frametitle{Test}

   \bi 
   \item Test each component of a program separately
   \item Use small subset of data
   \ei
\end{frame}

\begin{frame}
\frametitle{Reproducibility}

   \bi 
   \item All work should be reproducible
      \bi 
      \item Accidental deletion of results
      \item Computer/hard drive crashes
      \item Referee wants a slightly different specification but you can't find the .log file with the results
      \ei
   \ei
\end{frame}

\begin{frame}
\frametitle{Master file}

   \bi 
   \item In Stata, create a master .do file ('master.do') that contains:
      \bi 
      \item Descriptions of each step in analysis
      \item Calls to each .do file in order
      \ei
   \item Could run this file and reproduce all files and results
   \item Useful for documentation
      \bi 
      \item File structure
      \item History of analysis decisions
      \ei
   \item Can do something similar in Matlab, but Matlab's treatment of functions makes this a bit unnatural
   \ei
\end{frame}

\begin{frame}
\frametitle{Organization of files}

   \bi 
   \item Rule 1: One directory, and one directory only, for a project
      \bi 
      \item Some people use separate subdirectories for:
         \bi 
         \item Data files
         \item Program files (e.g., .sas, .m and .do files)
         \item Log files
         \ei
      \item I tend to keep them all in the same directory
      \ei
   \ei
\end{frame}

\begin{frame}
\frametitle{Organization of files}

   \bi 
   \item Rule 2: No interactive exploratory analysis is official until it's in a .do (.sas, .m) file
      \bi 
      \item Explore the data interactively, but anything that you really want to remember should be in a .do (.m) file
      \item The record of previously used commands makes this easy
      \ei
   \ei
\end{frame}

\begin{frame}
\frametitle{Organization of files}

   \bi 
   \item Rule 3: All .do (.sas, .m) files create logs
      \bi 
      \item The standard preamble always checks that any open logs are closed and a new log is started
      \item Do not have to be as protective of .log files as they can always be re-generated
         \bi 
         \item Unless of course the program takes awhile
         \ei
      \item In Matlab, use \mcode{diary} to create a log file
      	\bi
      	\item Downside: \texttt{diary} only appends and does not ever replace
      	\item As a result, I prefer to create Matlab log files through the \texttt{matsub} command
      	\ei
      \ei
   \ei
\end{frame}

\begin{frame}
\frametitle{Organization of files}

   \bi 
   \item Rule 4: Naming conventions can help in locating files
      \bi 
      \item Prefixes indicating purpose
         \bi 
         \item Creating data sets ('cr \emph{filename}.  \emph{ext}')
         \item Analysis files ('an \emph{filename}. \emph{ext}')
         \ei
      \item Use same name for log files as for program files (.m or.do)
      \ei
   \ei
\end{frame}

\begin{frame}
\frametitle{File naming conventions}

   \bi 
   \item Creating data sets
      \bi 
      \item When creating datasets, use the same name of the data for the program and log files
      \item Make sure to compress (pack) the data before storing to the server
      \ei
   \ei
\end{frame}

\begin{frame}
\frametitle{File naming conventions}

   \bi 
   \item Analysis files
      \bi 
      \item No creation of permanent datasets
      \item an*.ext files can be run in any order and will either:
         \bi 
         \item Produce the same results
         \item Not work at all
         \ei
      \item All datasets created within an*.ext files should be temporary
         \bi 
         \item Stata: tmpfile  \emph{lclname} [ \emph{lclname} [...]]
         \ei
      \ei
   \ei
\end{frame}

\begin{frame}
\frametitle{File naming conventions}

   \bi 
   \item Use suffixes that indicate order files
      \bi 
      \item Version number (\_v1)
      \item Date (\_20120708)
      \ei
   \item Common to have several files doing very similar things with slight changes
      \bi 
      \item Different estimation samples
      \item Different variable definitions
      \item Sensitivity analysis for different specifications
      \ei
   \ei
\end{frame}

\begin{frame}
\frametitle{Organization of files}

   \bi 
   \item Rule 5: Once a .do file is added to the master.do file, it is  \emph{never} edited again.
   \item If a change is necessary (e.g., add a variable to a specification), write a new .do (.sas, .m) file
   \item .do (.sas, .m) files are typically very small --> low memory cost to store many of them
   \ei
\end{frame}

\begin{frame}
\frametitle{Reproducibility}

   \bi 
   \item Side note: make sure that commands that are random are reproducible
      \bi 
      \item Set the random number seed before generating random number or drawing random samples (e.g., bootstrapping)
      \item In Stata, 'sort' will randomly break ties.
         \bi 
         \item Use 'isid  \emph{varlist}, sort' if existing variables uniquely identify an observation
         \item Create a unique identifier otherwise and include it in sort
         \ei
      \ei
   \ei
\end{frame}

\begin{frame}
\frametitle{Reproducibility}

   \bi 
   \item There are significant lags in publication process. You will likely have to return to code that is months old and understand it.
   \item Keep all programs (.sas, .m and .do) and .log files (these are small)
   \item Make sure all programs are well documented
   \ei
\end{frame}

\begin{frame}
\frametitle{Documenting Code}

   \bi 
   \item Why document? 
      \bi 
      \item Because you may write code and have to return to it 1 year later and alter it. 
      \item Those who do not document suffer a harsh punishment later...
      \ei
   \item The holy trinity of documentation in coding computer programs:
      \bi 
      \item Comments
      \item Meaningful Identifiers
      \item White Space
      \ei
   \ei
\end{frame}

\begin{frame}
\frametitle{Comments}

   \bi 
   \item Comment in such a way that someone else could understand what you're doing, including yourself months from now.
   \ei
\end{frame}

\begin{frame}
\frametitle{Commenting in Matlab}

   \bi 
   \item Begin with a \% and then write whatever you want 
   \item Multi-line comments can be offset with '\%\{' and '\%\}'
   		\bi
   		\item These are only useful when you are inside a continued line (see \url{http://blogs.mathworks.com/loren/2006/08/30/commenting-code/})
   		\ei
   \item Matlab's text editor will color these comments green
   \item Comments falling just beneath a function's header, i.e., [FunctionName]([Arguments]), are printed when one types: 
      \bi 
      \item help [FunctionName]
      \ei
   \ei
\end{frame}

\begin{frame}
\frametitle{Meaningful Identifiers}

   \bi 
   \item Variable names should clearly indicate what the variable does or is for
   \item One convention:
      \bi 
      \item GradSchoolIsFun
      \ei
   \item In Stata, variable and value labels make it easier to understand and data set
   \item Matlab doesn't have such utilities, so variable names are that much more important
   \ei
\end{frame}

\begin{frame}[fragile]
\frametitle{White Space}

   \bi 
   \item Use blank lines to separate your code into the appropriate (commented) blocks, e.g.
   \begin{lstlisting}
   %%%%%%%%%%%%%%%%%%%%%%%%
   % Initializing Variables
   %%%%%%%%%%%%%%%%%%%%%%%%
   \end{lstlisting}
   \item Use indentation of command nests (e.g., if-else-end, while-end, switch-case-end) 
      \bi 
      \item Matlab's text editor will perform these indentations automatically
      \ei
   \ei
\end{frame}

\begin{frame}[fragile]
\frametitle{White Space}
\bi
   \item Separate your code blocks into cells, e.g.
\begin{lstlisting}
%% Variable Initialization
educ = data(:,1);
...
%% OLS Estimation
...
\end{lstlisting}
   \item These cells can then be ``folded'' so that the structure of your program is immediately visible
   \item Whatever you put after the \%\% is automatically bolded
\ei
\end{frame}

\begin{frame}
\frametitle{White Space}

   \bi 
   \item Finally, wrap long command lines to ensure readability
      \bi 
      \item Matlab:  \mcode{...[enter]}
      \item Stata: 
         \bi 
         \item /* [enter]
         \item   */ 
         \item Can also reset delimiter (for a line) from 'enter' (default): ``\#delimit ;''
         \item /// [enter] will also do the trick
         \ei
      \ei
   \ei
\end{frame}

\begin{frame}
\frametitle{Network manners: Memory}

   \bi 
   \item Work with subsets when developing
   \item Compress data when saving
   \item Delete intermediary data files (you can always reproduce them exactly!)
   \item Zip files that are not in use
   \ei
\end{frame}

\begin{frame}
\frametitle{Network manners: CPU}

   \bi 
   \item Check server status before beginning long jobs
   \item Avoid looping over observations (i.e., vectorize code)
   \item Background jobs in batch mode
   \item 'nice' option in unix
   \ei
\end{frame}
\end{document}

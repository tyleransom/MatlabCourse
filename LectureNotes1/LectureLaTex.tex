\documentclass[english,xcolor=dvipsnames]{beamer}
% load package with ``framed'' and ``numbered'' option.
\usepackage[numbered,autolinebreaks,useliterate]{mcode}
\usepackage[orientation=landscape,size=custom,width=16,height=9,scale=0.48,debug]{beamerposter}
\usepackage[T1]{fontenc}
\usepackage[latin9]{inputenc}
\usepackage{amsthm}
\usepackage{amsmath}
\usepackage{amssymb}
\usepackage{bookmark}
\usepackage{verbatim}
\usepackage{graphics,graphicx}
\usepackage{pstricks,pst-node,pst-tree}
\usefonttheme{serif}
\usepackage{palatino}
%\usepackage[margin=.5cm]{geometry}

\definecolor{dgreen}{rgb}{0.,0.6,0.}
\definecolor{forest}{RGB}{34.,139.,34.}
\definecolor{byublue}{RGB}{0.,30.,76.}
\definecolor{dukeblue}{RGB}{0.,0.,156.}
%\usetheme{Ilmenau}
\usetheme{Warsaw}
\usecolortheme[named=dukeblue]{structure}
%\usecolortheme[named=RawSienna]{structure}
%\usecolortheme[named=byublue]{structure}
\setbeamertemplate{navigation symbols}{}
\setbeamertemplate{footline}{}
\setbeamercovered{transparent}

%%%%%%%%%%%%%%%%%%%%%%%%%%%%%%%%%%%%%%%%%%%%%%%%
% NOTE: With Ilmenau style, to get the bullets %
% looking right, do one section and one sub-   %
% section for each set of bullets              %
%%%%%%%%%%%%%%%%%%%%%%%%%%%%%%%%%%%%%%%%%%%%%%%%

\begin{document}
\begin{frame}
\title{Introduction to \LaTeX}
\author{
	Tyler Ransom\\
	\emph{Duke University}\\
%    \today \\
%    \vspace{10cm}
}
\titlepage
\end{frame}

\begin{frame}{Outline}
\begin{itemize}
	\item What is \LaTeX?
	\item How does it work?
	\item Why should I use it?
\end{itemize}
\end{frame}

\begin{frame}{Main Components of \LaTeX}
\begin{description}
	\item[Typesetting engine:] MiKTeX  for PC; MacTeX  for Mac; TeX Live for Unix
	\item[Source code file:] A text file (.tex) that contains commands---analagous to a .m file for Matlab or .do file for Stata
	\item[Compiler:] A program that tells the typesetting engine to process your source code file(s)
\end{description}
\end{frame}

\begin{frame}{How to get from source to PDF}
\begin{itemize}
	\item Most compilers (e.g. LyX, TeXnicCenter, TeXworks) integrate a source code text editor for ease of processing
	\item LyX provides a What-You-See-Is-What-You-Get interface that makes things much simpler
	\item \texttt{pdflatex} is the typesetting engine's command to convert your .tex source file to a PDF
	\item Your compiler of choice will typically have some button that executes a command to \texttt{pdflatex} 
\end{itemize}
\end{frame}

\begin{frame}{Anatomy of a source code file}
A \LaTeX source file is broken up into two main sections
\begin{enumerate}
	\item Preamble (where user declares which packages and global options will be used)
	\item Document (the content of your document) 
\end{enumerate}
\end{frame}

\begin{frame}[fragile]{Example}
\begin{verbatim}
\documentclass{article}
\title{The Pythagorean Theorem Revisited}
\author{Nicholas Halden}
\date{July 31, 2011}

\begin{document}
   \maketitle
   The Pythagorean Theorem states that $a^{2}+b^{2}=c^{2}.$
\end{document}
\end{verbatim}
\end{frame}

\begin{frame}{Power of \LaTeX}
\begin{itemize}
	\item Documents produced in \LaTeX\,are typeset---meaning there is no need for the user to worry about spacing between paragraphs, margins, etc.---the typesetting engine takes care of where to optimally place objects
	\item Academic publishers use \LaTeX\,to typeset books and journals, so your product can look just as professional
	\item Math equations look especially nice
	\item Automatic export to PDF, a universally viewable file format
	\item Millions of users, so if you run into a problem, Google can easily find a solution
	\begin{itemize}
		\item Note: I have never found a problem that someone else hasn't already encountered and solved on some discussion board somewhere
	\end{itemize}
\end{itemize}
\end{frame}

\begin{frame}{Drawbacks of \LaTeX}
\begin{itemize}
	\item Somewhat of a learning curve (though this is greatly mitigated with WYSIWYG programs like LyX and Scientific Workplace)
	\item Typesetter may not always put objects where you want them
	\item Can be finicky if you try to get too fancy
\end{itemize}
\end{frame}

\begin{frame}{Typesetting your first \LaTeX\,document}
\begin{itemize}
	\item For beginners, I highly recommend LyX---it works on all platforms, doesn't require you to know any \LaTeX\,commands, and is already installed on all of the Econ department machines in the Bowling Alley and the Driving Range.
	\item LyX's WYSIWYG structure allows the user to see updates to his document without having to re-compile.
	\item For more advanced users, LyX has limitations in what it can do.
	\item LyX has great documentation and tutorials for first-timers. When first opening LyX go to \textsf{Help$\blacktriangleright$Introduction} and \textsf{Help$\blacktriangleright$Tutorial}. Nothing else I can say will help you more than going through the Introduction and Tutorial on your own.
	\item We'll go through this today.
\end{itemize}
\end{frame}

\begin{frame}{Anatomy of a \LaTeX\,document: Environments}
\begin{itemize}
	\item \LaTeX\, has what are called \emph{environments}
	\item These include things like:
	\begin{itemize}
		\item Section headings
		\item Section subheadings
		\item Numbered lists
		\item Mathematical equations
		\item Quotations
		\item Normal text
	\end{itemize}
	\item In LyX, you can access these in the top left (just below the `File' menu)
\end{itemize}
\end{frame}

\begin{frame}{Anatomy of a \LaTeX\,document: Document Classes}
\LaTeX\, has different document classes which organize content in different ways:
\begin{itemize}
	\item article (scientific journal article; the default class)
	\item book (allows for chapters and front/back matter)
	\item presentation (presentation slides)
	\item letter (includes extra environments for address lines, signature, etc.)
\end{itemize}
In LyX, change the document class by going to Document$\blacktriangleright$Settings
\end{frame}

\begin{frame}{Anatomy of a \LaTeX\,document: Labels and Cross-References}
One of the great things about \LaTeX\, is that users can create ``links'' within a document to other parts of the document. e.g. if I referenced a equation 1 on page 5, I can create a link that the reader can click on and go back to page 3 where equation 1 was first defined. The following items can be referenced:
\begin{itemize}
	\item mathematical equations
	\item chapters and sections and subsections (and subsubsections and ...)
	\item footnotes
	\item bibliographic references
\end{itemize}
\end{frame}

\begin{frame}{Anatomy of a \LaTeX\,document: Reference Sections}
\LaTeX\, also makes tables of contents, bibliographies and indexes very easy to generate. 
\begin{itemize}
	\item Bibliographies are managed through Bib\TeX
	\item Bib\TeX can accommodate any citation style (e.g. MLA, APA, Chicago, etc.)
	\item Most professional journals require a Bib\TeX database of your references before publication
	\item Tables of contents and indexes are typically only used in books (not journal articles)
\end{itemize}
\end{frame}

\begin{frame}[fragile]{Anatomy of a \LaTeX\,document: Math Formulas}
\LaTeX\, was primarily invented to handle mathematical formulas
\begin{itemize}
	\item Many commands for many types of formulas
	\item Greek letters invoked with, e.g. \begin{verbatim}{\beta}\end{verbatim}
	\item Can do equations, matrices, fractions, accent marks (like $\tilde{x}$ and $\hat{y}$)
	\item Other crazy stuff like
\end{itemize}
\begin{equation*}
a_0+\cfrac{b_1}{
a_1+\cfrac[l]{b_2}{
a_2+\cfrac[r]{b_3}{
a_3+\cdots}}}
\end{equation*}
\end{frame}


\begin{frame}{Floats, Figures, Tables, Graphics}
\begin{itemize}
	\item Tables and figures look best when inserted in a float
	\item Graphics images need to be certain file types: .jpg, .eps, .png
	\item Can't use .gif
	\item \LaTeX\,automatically numbers the floats
	\item It's also possible to create sub-figures and sub-tables that are numbered (a), (b), etc.
\end{itemize}
\end{frame}

\begin{frame}{Exporting/Importing \LaTeX\,documents into LyX}
\begin{itemize}
	\item You can easily import into LyX a .tex document that someone else has written
	\item File$\blacktriangleright$Import$\blacktriangleright$LaTeX (plain)
	\item Exporting a .lyx document into a .tex document (for someone else to read) is just as easy:
	\item File$\blacktriangleright$Export$\blacktriangleright$LaTeX (plain)
\end{itemize}
\end{frame}

\begin{frame}{Beamer}
\begin{itemize}
	\item Beamer is the ``PowerPoint'' of \LaTeX
	\item These slides were made in Beamer
	\item To use Beamer in LyX, select article(beamer) as the document class
	\item For a tutorial in Beamer, go to \url{http://www.uncg.edu/cmp/reu/presentations/Charles\%20Batts\%20-\%20Beamer\%20Tutorial.pdf}
	\item For a graphical table of Beamer slide styles, go to \url{http://www.hartwork.org/beamer-theme-matrix/}
\end{itemize}
\end{frame}

\end{document}
\documentclass[english,xcolor=dvipsnames]{beamer}
% load package with ``framed'' and ``numbered'' option.
\usepackage[autolinebreaks,useliterate]{mcode}
\usepackage[orientation=landscape,size=custom,width=16,height=9,scale=0.48,debug]{beamerposter}
\usepackage[T1]{fontenc}
\usepackage[latin9]{inputenc}
\usepackage{amsthm}
\usepackage{amsmath}
\usepackage{amssymb}
\usepackage{bookmark}
\usepackage{verbatim}
\usepackage{graphics,graphicx}
\usepackage{pstricks,pst-node,pst-tree}
\usefonttheme{serif}
\usepackage{palatino}
%\usepackage[margin=.5cm]{geometry}

\definecolor{dgreen}{rgb}{0.,0.6,0.}
\definecolor{forest}{RGB}{34.,139.,34.}
\definecolor{byublue}{RGB}{0.,30.,76.}
\definecolor{dukeblue}{RGB}{0.,0.,156.}
%\usetheme{Ilmenau}
\usetheme{Warsaw}
\usecolortheme[named=dukeblue]{structure}
%\usecolortheme[named=RawSienna]{structure}
%\usecolortheme[named=byublue]{structure}
\setbeamertemplate{navigation symbols}{}
\setbeamertemplate{footline}{}
\setbeamercovered{transparent}

%%%%%%%%%%%%%%%%%%%%%%%%%%%%%%%%%%%%%%%%%%%%%%%%
% NOTE: With Ilmenau style, to get the bullets %
% looking right, do one section and one sub-   %
% section for each set of bullets              %
%%%%%%%%%%%%%%%%%%%%%%%%%%%%%%%%%%%%%%%%%%%%%%%%

\begin{document}
\begin{frame}
\title{Intro to Bayesian Inference\thanks{Based on Greene Chapter 18}}
\author{
	Tyler Ransom\\
	\emph{Duke University}\\
%    \today \\
%    \vspace{10cm}
}
\titlepage
\end{frame}

\begin{frame}{Outline}
\begin{itemize}
	\item Introduce Bayesian approach
	\item Compare/Contrast Bayesian vs. Frequentist approaches
	\item Discuss Random Coefficients
\end{itemize}
\end{frame}

\begin{frame}{Bayesian Approach}
\begin{itemize}
	\item Consider a simple problem, viewed through the lens of both classical statistics and Bayesian statistics (adapted from Wikipedia):
	\item There are two bags of marbles. Bag \#1 has 10 red and 10 blue. Bag \#2 has 20 red and 10 blue.
	\item \textbf{Classical statistics}: What is the probability that a marble drawn at random from one of the bags is blue?
	\item \textbf{Bayesian statistics}: Given that the marble drawn is blue, what is the probability that it was drawn from Bag \#1?
\end{itemize}
\end{frame}

\begin{frame}{Bayesian Approach}
\begin{itemize}
	\item The classical approach would say something like ``assume each bag can be drawn from with equal probability,'' and in the end we recover a probability of drawing a blue marble 
	\item The Bayesian approach would say something like ``assume each bag can be drawn from with equal probability,'' (a prior belief) and in the end, we recover a posterior probability that the draw came from Bag \#1
\end{itemize}
\end{frame}

\begin{frame}{Bayesian MLE}
Recall Bayes' rule:
\[
P(A\,\vert\,B) = \frac{P(B\,\vert\,A)P(A)}{P(B)}
\]
We can rewrite this as
\[
P(\textrm{parameters}\,\vert\,\textrm{data}) = \frac{P(\textrm{data}\,\vert\,\textrm{parameters}) P(\textrm{parameters})}{P(\textrm{data})}
\]
or, dividing out $P(\textrm{data})$ (which is uninformative for our purposes)
\[
P(\textrm{parameters}\,\vert\,\textrm{data}) \propto P(\textrm{data}\,\vert\,\textrm{parameters}) P(\textrm{parameters})
\]
or
\[
\textrm{posterior\,density} = \textrm{likelihood\,function}\times\textrm{prior\,density}
\]
\end{frame}

\begin{frame}{Bayesian vs. Classical Approach}
\begin{itemize}
	\item Bayesian proponents say that estimation is not deducing values of fixed parameters, but rather sharpening subjective beliefs about the state of the world
	\item But if this subjectivity is uncomfortable because a researcher prefers ``black and white'' truths, then classical method may be preferable
	\item Basically, each approach has pros and cons
\end{itemize}
\end{frame}

\begin{frame}{Another Example}
Another example to illustrate the Bayesian way of thinking: Consider the problem of predicting the temperature on a certain day of the year
\begin{itemize}
	\item Classical statistics would possibly estimate an OLS model of temperature on a variety of covariates and then generate a prediction
	\item Bayesian statistics would create a prior distribution based on (e.g. the historical average of the temperatures on that day) and then estimate a posterior distribution
\end{itemize}
\end{frame}

\begin{frame}{Pros and Cons of Bayesian Approach}
Pros
\begin{itemize}
	\item Can find efficient solutions to complex estimation problems (e.g. MCMC, which is quite popular)
	\item Doesn't tie the researcher to a single ``black and white'' result
	\item Recognizes that research process is sharpening of beliefs, not finding the one right answer
\end{itemize}
Cons
\begin{itemize}
	\item More difficult to interpret results
	\item Requires more parameterization than classical MLE
	\item Posterior is influenced by the prior (so if your prior is bad, posterior may also be bad)
\end{itemize}
\end{frame}

\begin{frame}{Random coefficients}
\begin{itemize}
	\item Random coefficients refers to a modeling technique where each individual is assumed to have a different $\beta$, e.g. $\beta_{i} \sim N(\beta,\Sigma)$
	\item Since the Bayesian approach assumes a distribution of the parameter vector, it is tempting to equate the posterior distribution with the distribution of random coefficients
	\item The two methods are quite different, and although both involve a distribution, they should not be confused
\end{itemize}
\end{frame}

\begin{frame}{Random Coefficients vs. Unobserved Heterogeneity}
When should I use cluster-corrected standard errors, and when should I use robust standard errors?
\begin{itemize}
	\item We previously talked about heterogeneity in unobservables, which allows us to recover ``correct'' theta estimates
	\item Introducing heterogeneity into the parameter estimates is another method of allowing for ``correct'' estimates; it's just harder to interpret when you get a different coefficient vector for each person
	\item Now need to make distributional assumptions on theta AND the error term
\end{itemize}
\end{frame}

\begin{frame}{Random Coefficients: Example}
This example comes from Train (pp. 137-138)
\begin{itemize}
	\item Suppose the utility for individual $i$ for choice $j$ is $u_{ij} = X_{ij}\beta_{i}+\varepsilon_{ij}$
	\item The choice probability $P_{ij}$ is now an integral over the (continuous) distribution of $\beta$:
	\[
	P_{ij} = \int \left(\frac{\exp(X_{ij}\beta)}{\sum_{k} \exp(X_{ik}\beta)}\right) f(\beta) d\beta
	\]
	\item So our likelihood function is now
	\[
	\ell = \sum_{i} \sum_{j} y_{ij}\ln \left(P_{ij}\right)
	\]
	where the maximization goes over the parameters of the distribution (e.g. $\beta$ and $\Sigma$)
\end{itemize}
\end{frame}


\end{document}
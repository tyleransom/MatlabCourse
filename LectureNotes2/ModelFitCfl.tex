\documentclass[english,xcolor=dvipsnames]{beamer}
% load package with ``framed'' and ``numbered'' option.
\usepackage[autolinebreaks,useliterate]{mcode}
\usepackage[orientation=landscape,size=custom,width=16,height=9,scale=0.48,debug]{beamerposter}
\usepackage[T1]{fontenc}
\usepackage[latin9]{inputenc}
\usepackage{amsthm}
\usepackage{amsmath}
\usepackage{amssymb}
\usepackage{bookmark}
\usepackage{verbatim}
\usepackage{graphics,graphicx}
\usepackage{pstricks,pst-node,pst-tree}
\usefonttheme{serif}
\usepackage{palatino}
%\usepackage[margin=.5cm]{geometry}

\definecolor{dgreen}{rgb}{0.,0.6,0.}
\definecolor{forest}{RGB}{34.,139.,34.}
\definecolor{byublue}{RGB}{0.,30.,76.}
\definecolor{dukeblue}{RGB}{0.,0.,156.}
%\usetheme{Ilmenau}
\usetheme{Warsaw}
\usecolortheme[named=dukeblue]{structure}
%\usecolortheme[named=RawSienna]{structure}
%\usecolortheme[named=byublue]{structure}
\setbeamertemplate{navigation symbols}{}
\setbeamertemplate{footline}{}
\setbeamercovered{transparent}

%%%%%%%%%%%%%%%%%%%%%%%%%%%%%%%%%%%%%%%%%%%%%%%%
% NOTE: With Ilmenau style, to get the bullets %
% looking right, do one section and one sub-   %
% section for each set of bullets              %
%%%%%%%%%%%%%%%%%%%%%%%%%%%%%%%%%%%%%%%%%%%%%%%%

\begin{document}
\begin{frame}
\title{Model Fit \& Counterfactual Simulations}


\author{Tyler Ransom}


\institute{Duke University}

\titlepage

\end{frame}


\begin{frame}{Model Fit}

Model fit is an important dimension by which researchers gauge the
quality of their work
\begin{itemize}
\item Particularly in structural work, where the model can at times do more
of the talking than the data, assessing model fit is a requirement
\item At the end of the day a researcher needs to demonstrate that her model
reasonably approximates the real world
\end{itemize}
\end{frame}

\begin{frame}{Model Fit}
In a linear model, assessing model fit is quite simple:
\begin{itemize}
\item Generate $\hat{y}=X\hat{\beta}$, then compare $y$ and $\hat{y}$
\item $R^{2}$ a goodness-of-fit measure (but shouldn't be used to compare
competing models---i.e. don't maximize $R^{2}$)
\end{itemize}
In a nonlinear model, where the dependent variable is not continuous,
assessing model fit may not be quite as straightforward
\begin{itemize}
\item Discrete choice model: Match $\hat{P}_{j}$ with $\left(1/N\right)\sum_{i}1\left[y=j\right]$,
i.e. compare what the model predicts for the choice probabilities
with what the data say
\item Other nonlinear models: Match the average of some measure of the dependent
variable with some average measure from data
\end{itemize}

\end{frame}


\begin{frame}{Model Fit}
\begin{itemize}
\item Nonlinear analog to $R^{2}$ is called the Likelihood Ratio Index
(LRI):
\[
LRI=1-\frac{\ln L}{\ln L_{0}}
\]
where $\ln L$ is the log likelihood for the full model and $\ln L_{0}$
is the log likelihood for the model with only an intercept.
\item Other measures of goodness-of-fit are the Akaike Information Criterion
(AIC) and the Bayes-Schwarz Information Criterion (BIC)

\begin{itemize}
\item These measures are similar to adjusted $R^{2}$ in that they penalize
the inclusion of more variables
\end{itemize}
\item $BIC=-2\ln L+k\ln\left(n\right)$, where $k$ is the number of estimated
parameters and $n$ is the sample size
\item $AIC=2k-2\ln L$
\end{itemize}

\end{frame}


\begin{frame}{Counterfactual Simulation}
\begin{itemize}
\item Counterfactual simulation is the method by which structural empiricists
simulate policy changes
\item The main payoff to the structural approach (i.e. making a ton of assumptions---some
more realistic than others) is that researchers can turn off certain
channels of their model and literally see the effects of policy changes
\item To the extent that the model is a good approximation to the real world,
these counterfactual simulations can be quite enlightening
\item A good structural model takes into account both direct and indirect
effects (see below)
\item The counterfactual simulation is able to account for what happens
to both direct and indirect effects
\end{itemize}

\end{frame}


\begin{frame}{Counterfactual Simulation Example 1}

Let's look at how counterfactual simulation answers a research question
in a discrete choice model (recreational choice)
\begin{itemize}
\item In this model, people choose from among $J$ lakes to go fishing/boating/swimming
in, or choose not to go at all
\item People are utility maximizers, so the choice we observe is their utility
maximizing choice (i.e. this is a random utility model)
\item The following variables enter the utility function: travel distance,
travel cost, park entry fees, lake amenities, fishing conditions,
and congestion (how many other people are at the recreation site)
\item The research question is ``what are the environmental implications
of (i.e. what happens to park attendance in response to) an entry
fee subsidy or a gas tax?''
\end{itemize}

\end{frame}


\begin{frame}{Counterfactual Simulation Example 1}

We can answer our research question by completing the following steps:
\begin{itemize}
\item Simulate the effects of a gas tax by increasing travel costs for each
person (proportional to travel distance)
\item Simulate the effects of an entry fee subsidy by decreasing entry fees
for each person
\item Given these new values of the $X$ variables, re-generate the $\hat{P}_{j}$
for each location
\item Observe how $\hat{P}_{j,0}$ (the baseline choice probabilities) differ
from $\hat{P}_{j,1}$ (the post-simulation choice probabilities)
\item From this, the researcher can back out how many people go to each
lake under the new scenario
\item This can be informative of, e.g., the price elasticity of demand for
lake recreation
\end{itemize}

\end{frame}


\begin{frame}{Counterfactual Simulation Example 2}

Let's consider a model of job search
\begin{itemize}
\item Person chooses to accept a job (if she receives an offer) based on
the offered wage, offered hours/week, and offered health care benefits
\item State variables are her own health status, the health status of her
spouse, and the health care benefits from her spouse's job
\item She accepts the job if her utility from accepting the offer is larger
than her utility from rejecting the offer
\item Now a researcher can simulate the effects of a health care policy
change (e.g. insurance can be obtained cheaply while unemployed)
\item What happens to the reservation wage of the workers?
\item What happens to the duration of unemployment spells?
\end{itemize}

\end{frame}


\begin{frame}{Conclusion}
\begin{itemize}
\item Your econometric model should do a good job of fitting the data
\item There are various metrics by which to gauge model fit
\item Counterfactual simulation is a great way to answer economic questions
\item When doing work with structural models, your reasearch question is
not answered until you've completed the counterfactual simulation(s)
\end{itemize}

\end{frame}
\end{document}

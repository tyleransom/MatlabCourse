%% LyX 2.2.3 created this file.  For more info, see http://www.lyx.org/.
%% Do not edit unless you really know what you are doing.
\documentclass[12pt,english]{article}
\usepackage{mathptmx}
\usepackage[T1]{fontenc}
\usepackage[latin9]{inputenc}
\usepackage{geometry}
\geometry{verbose,tmargin=1in,bmargin=1in,lmargin=1in,rmargin=1in}
\usepackage{babel}
\usepackage[authoryear]{natbib}
\usepackage[unicode=true,pdfusetitle,
 bookmarks=true,bookmarksnumbered=false,bookmarksopen=false,
 breaklinks=false,pdfborder={0 0 0},pdfborderstyle={},backref=false,colorlinks=false]
 {hyperref}
\usepackage{breakurl}

\makeatletter

%%%%%%%%%%%%%%%%%%%%%%%%%%%%%% LyX specific LaTeX commands.
%% Because html converters don't know tabularnewline
\providecommand{\tabularnewline}{\\}

%%%%%%%%%%%%%%%%%%%%%%%%%%%%%% User specified LaTeX commands.
\date{}

\makeatother

\begin{document}

\title{\textbf{Matlab for Applied Micro Empiricists: }\\
\textbf{Summer II.2 2012}}

\maketitle
\vspace{-0.85in}

\begin{center}
\begin{tabular}{ll}
Course: & Matlab for Applied Micro Empiricists (half Summer module for rising
second-years)\tabularnewline
Instructor: & Tyler Ransom\tabularnewline
Time: & Tue/Thu 10 a.m. - 12:30 p.m.; Friday 11 a.m. - 12 p.m.\tabularnewline
Location: & Social Sciences 113\tabularnewline
Office: & 2106 Campus Dr \#201A\tabularnewline
Email: & \href{mailto:tyler.ransom@duke.edu}{tyler.ransom@duke.edu}\tabularnewline
Office Hours: & By appointment\tabularnewline
\end{tabular}
\par\end{center}

\subsubsection*{About this course}

This module will build on the skills introduced in the first module
and introduce topics covered in second-year applied micro modules.
These topics include multinomial choice models, numerical integration,
Monte Carlo simulation and bootstrapping. The course will cover specific
applications of these topics in the fields of labor, education and
industrial organization.

\subsubsection*{Prerequisites}

I am going to assume that all students know the material from the
first Matlab module. This includes familiarity with Matlab basics
(most notably Matlab's functional optimizers) as well as a working
knowledge of \LaTeX.

\subsubsection*{Textbooks}

There is no formal textbook for the course, but students may find
the following resources helpful: \emph{Discrete Choice Methods with
Simulation} (Train)\footnote{This is available for free online at \url{http://elsa.berkeley.edu/books/choice2.html}},
\emph{Econometric Analysis of Cross Section and Panel Data} (Wooldridge),
\emph{Econometrics} (Hayashi), \emph{Time Series Analysis} (Hamilton),
\emph{Econometric Analysis} (Greene), and \emph{Numerical Methods
in Economics} (Judd).

\subsubsection*{Registration, Enrollment and Overall Class Grades}

In order to get credit for this course, you need to first enroll in
Econ 360 or Econ 370 in ACES. Because each module is not listed separately
in ACES, I can only determine your enrollment in this course through
your completion of problem sets and/or attendance in lectures. In
order to get credit for Econ 360 or Econ 370 in ACES, each student
must enroll in at least two modules. Grades from those two modules
will be averaged into a final grade for Econ 360/370. If you enroll
in more than two modules, your two highest grades are averaged. Because
of this favorable grading policy, you are encouraged to take more
than two modules.

\subsubsection*{Grades for this Module}

Grades will be determined by the average score from three problem
sets, due each week by 11:59 p.m. on Thursday. Late problem sets will
not be accepted. Submit problem set materials to your ``dropbox''
folder on Sakai. You are allowed to work on problem sets in groups
(no larger than 3, please), but each student must turn in his/her
own copy of the problem set. In particular, each student should avoid
copying/pasting code and instead type the code out on his/her own.
(This is the only way to learn how to program.) Put your name and
the names of those in your group at the top of your code file(s).
Each Friday morning I will post solutions at approximately 8 A.M.
We will then spend the Friday lecture time going through the code
together. Problem sets will be graded on the following scale (some
convex combination of effort and accuracy):

\medskip{}
\begin{center}
\begin{tabular}{cl}
4: & Problem set is complete and mostly correct\tabularnewline
3: & Problem set is complete with errors; or mostly complete and mostly
correct\tabularnewline
2: & Problem set is complete with many errors; or barely complete and mostly
correct\tabularnewline
1: & Problem set is barely attempted or completely incorrect\tabularnewline
0: & Problem set turned in late or not at all\tabularnewline
\end{tabular}\\
\par\end{center}

Problem set grades will be combined to an unweighted average to determine
course grade.

\subsubsection*{Schedule of Topics (subject to change)}
\begin{center}
\begin{tabular}{ccll}
\hline 
Class & Date & ~~~~~~~~~~~~~~~~~~~~~~~~~~~~~~Topics & Lecture Notes\tabularnewline
\hline 
1 & Tue 7/24 & Intro to multinomial choice & Multinomial.pdf\tabularnewline
 &  & Unobserved heterogeneity & MultinomialDerivations.pdf\tabularnewline
 &  & Fixed effects/random effects & UnobservedHeterogeneity.pdf\tabularnewline
2 & Thu 7/26 & Advanced programming tricks & TipsAndTricks.pdf\tabularnewline
 &  & Constrained optimization & DeltaMethod.pdf\tabularnewline
 &  & \emph{PS1 due by 11:59 p.m.} & \tabularnewline
3 & Fri 7/27 & Go over code for Problem Set 1 & \tabularnewline
4 & Tue 7/31 & Improving computational efficiency & ComputationalEfficiency.pdf\tabularnewline
 &  & Analytical gradients & LinuxLab.pdf\tabularnewline
 &  & mex files and compiled languages & \tabularnewline
5 & Thu 8/2 & Simulation & NumericalIntegration.pdf\tabularnewline
 &  & Integration & Bootstrap.pdf\tabularnewline
 &  & Bootstrapping/Inference & \tabularnewline
 &  & \emph{PS2 due by 11:59 p.m.} & \tabularnewline
6 & Fri 8/3 & Go over code for Problem Set 2 & \tabularnewline
7 & Tue 8/7 & Multi-stage estimation algorithms & IterativeAlgorithms.pdf\tabularnewline
 &  & Intro to Bayesian Inference & IntroBayesianInference.pdf\tabularnewline
8 & Thu 8/9 & Model fit & ModelFitCfl.pdf\tabularnewline
 &  & Counterfactual simulation & DurationCount.pdf\tabularnewline
 &  & Duration Analysis & \tabularnewline
 &  & Count Data Models & \tabularnewline
 &  & \emph{PS3 due by 11:59 p.m.} & \tabularnewline
9 & Fri 8/10 & Go over code for Problem Set 3 & \tabularnewline
\hline 
\end{tabular}
\par\end{center}
\end{document}

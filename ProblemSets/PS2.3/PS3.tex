%% LyX 2.0.0 created this file.  For more info, see http://www.lyx.org/.
%% Do not edit unless you really know what you are doing.
\documentclass[12pt,english]{article}
\usepackage{mathptmx}
\usepackage[T1]{fontenc}
\usepackage[latin9]{inputenc}
\usepackage{geometry}
\geometry{verbose,tmargin=1in,bmargin=1in,lmargin=1in,rmargin=1in}
\usepackage{babel}
\usepackage{esint}
\usepackage[unicode=true,pdfusetitle,
 bookmarks=true,bookmarksnumbered=false,bookmarksopen=false,
 breaklinks=false,pdfborder={0 0 0},backref=false,colorlinks=false]
 {hyperref}
\usepackage{breakurl}

\makeatletter
%%%%%%%%%%%%%%%%%%%%%%%%%%%%%% User specified LaTeX commands.
\date{}

\makeatother

\begin{document}

\title{Problem Set 3}

\maketitle
Directions: Answer all questions. Clearly label all answers. Show
all of your code. Compute standard errors for all parameter estimates.
Turn in the following to me via your dropbox (in a folder labeled
'MatlabPS2.3') in Sakai by 11:59 p.m. on Thursday, August 9, 2012: 
\begin{itemize}
\item m-file(s)
\item .sh shell script file that executes your m-file(s)%
\footnote{Name your job something other than {}``matsub,'' and have it send
you an email upon completion.%
}
\item a log file (from off the cluster)
\item shell\_name.oXXXXX file
\item pdf version of your writeup with its \LaTeX source code
\end{itemize}
Put the names of all group members at the top of your writeup (each
student must turn in his/her own materials). As usual, set the seed
for \texttt{rand} and \texttt{randn} to 1234.
\begin{enumerate}
\item Simulation of a definite integral with no closed form. In this question,
compute the integral of $f\left(x\right)$ from $-10$ to $1$, where
\[
f\left(x\right)=\exp\left(-x^{2}\right)
\]
This integral has no closed form solution, so we will approximate
its value by two methods:

\begin{enumerate}
\item Use one of the quadrature formulas in Matlab to approximate the solution.
Note: it's easiest (but not necessary) to use a function handle to
define the function.
\item Simulate the solution to the integral using the notes from the lecture
slide entitled {}``general steps for simulating an integral.'' Compare
your solution with different numbers of draws. How many draws are
required to get close to the solution that the quadrature gave? How
long did each procedure take?
\end{enumerate}
\item Bootstrapping, using the dataset \texttt{nlsw88.mat} (from problem
set 1 of the first Matlab module).

\begin{enumerate}
\item Bootstrap the sample mean of the variable \texttt{ttl\_exp} using
Matlab's \texttt{bootstrp} function. Do 10,000 bootstrap iterations.
How does the boostrapped mean compare to the population mean? How
does the standard error of the bootstrapped mean compare to its population
analog?%
\footnote{Recall that the standard error of the mean of a variable is $\sigma/\sqrt{n}$,
where $\sigma$ is the standard deviation.%
}
\item Bootstrap the sample median of the natural logarithm of the variable
\texttt{wage} without using Matlab's \texttt{bootstrp} function. Do
10,000 bootstrap iterations. How does the boostrapped median compare
to the population median? How does the standard error of the bootstrapped
median compare to its population analog?%
\footnote{The standard error of the median of a variable is $1.253\sigma/\sqrt{n}$,
where $\sigma$ is the standard deviation.%
}
\item Estimate the following model using OLS (same as problem set 2 of the
first Matlab module). Feel free to use any estimation method available
to you.
\begin{eqnarray*}
\ln\left(wage_{i}\right) & = & \beta_{1}+\beta_{2}age_{i}+\beta_{3}black_{i}+\beta_{4}other_{i}+\beta_{5}collgrad_{i}+\\
 &  & \beta_{6}grade_{i}+\beta_{7}married_{i}+\beta_{8}south_{i}+\beta_{9}c\_city_{i}+\\
 &  & \beta_{10}union_{i}+\beta_{11}ttl\_exp_{i}+\beta_{12}tenure_{i}+\beta_{13}age_{i}^{2}+\\
 &  & \beta_{14}hours_{i}+\beta_{15}never\_married_{i}+\varepsilon_{i}
\end{eqnarray*}
(Be sure to drop observations for all variables where any of the variables
are missing.) Obtain bootstrapped standard errors for 10,000 replications
for this simple OLS model using Matlab's \texttt{bootstrp} function.
How do your bootstrapped $\beta$'s and standard errors compare to
the $\beta$'s and standard errors from the regular OLS?
\item Repeat (c) but this time code the bootstrapping by hand. Do not use
loops over individual observations (or else your bootstrap will never
finish).
\end{enumerate}
\item Examining heterogeneity methods in a binomial probit model. Using
the dataset \texttt{nlsy97.mat} (from problem set 1 of this module),
consider the following model (similar to before)
\begin{eqnarray*}
d_{it} & = & \beta_{1}+\beta_{2}male_{i}+\beta_{3}AFQT_{i}+\beta_{4}Mhgc_{i}+\beta_{5}hgc_{it}+\beta_{6}exper_{it}+\\
 &  & \beta_{7}Diploma_{it}+\beta_{8}AA_{it}+\beta_{9}BA_{it}+\alpha_{i}+\varepsilon_{it}
\end{eqnarray*}
where $d_{it}$ is 1 if $activity_{it}=2$ and 0 otherwise. Assume
a random effect: $\alpha_{i}\sim Bernoulli\left(\pi\right)$, where
$\alpha_{i}=0$ with probability $1-\pi$, $\alpha_{i}=1$ with probability
$\pi$, and $\varepsilon_{it}\sim N\left(0,1\right)$. 

\begin{enumerate}
\item First estimate the model with no heterogeneity using \texttt{glmfit},
i.e. assume $\alpha_{i}=0$ for all $i$.
\item Estimate the model with unobserved heterogeneity using \texttt{fminunc}.%
\footnote{You may need to use \texttt{fminsearch}, then start \texttt{fminunc}
at the \texttt{fminsearch} answers. This is an example of when optimization
can be more of an art than a science.%
} Use the \texttt{glmfit} estimates + random noise as your starting
values. How do your parameter estimates change? Report the standard
errors for each of your parameters. What is $\hat{\pi}$? How would
you interpret this coefficient estimate?
\item Now assume that the random effect follows a normal distribution; i.e.
$\alpha_{i}\sim N\left(0,\sigma_{\alpha}^{2}\right)$. Estimate the
parameters $\beta$ and $\sigma_{\alpha}^{2}$ using the Composite
Simpson's rule%
\footnote{See {}``Composite Simpson's rule'' in Wikipeida for further details.%
} over discretized values of $\alpha_{i}$ (e.g. from -3 to 3). In
order to integrate out the random effects, your log likelihood function
should look something like
\[
\ell=\sum_{i}\ln\left(\int_{-3}^{3}g_{i}\left(\alpha_{i}\right)f\left(\alpha_{i}\right)d\alpha_{i}\right)
\]
 where
\[
g_{i}\left(\alpha_{i}\right)=\prod_{t}\Phi\left(X_{it}\beta+\alpha_{i}\right)^{y_{it}}\left(1-\Phi\left(X_{it}\beta+\alpha_{i}\right)\right)^{1-y_{it}}
\]
and $f\left(\alpha_{i}\right)$ is the pdf of the normal distribution
with mean $0$ and variance $\sigma_{\alpha}^{2}$, evaluated at $\alpha_{i}$.
The Composite Simpson's rule for quadrature states that the integral
can be approximated as
\begin{eqnarray*}
\int_{-3}^{3}g_{i}\left(\alpha_{i}\right)f\left(\alpha_{i}\right)d\alpha_{i} & \approx & \frac{h}{3}\left[g_{i}\left(-3\right)f\left(-3\right)+4g_{i}\left(-3+h\right)f\left(-3+h\right)+2g_{i}\left(-3+2h\right)f\left(-3+2h\right)\right.\\
 &  & \left.\,\,\,\,+\cdots+4g_{i}\left(3-h\right)f\left(3-h\right)+g_{i}\left(3\right)f\left(3\right)\right],
\end{eqnarray*}
where $h$ is the step size of the discretized grid. Hence, the log
likelihood function should look like
\[
\ell=\sum_{i=1}^{n}\ln\left(\sum_{g=1}^{G}\left\{ \prod_{t=1}^{T}\Phi\left(X_{it}\beta+\alpha_{g}\right)^{y_{it}}\left(1-\Phi\left(X_{it}\beta+\alpha_{g}\right)\right)^{1-y_{it}}\right\} s_{g}w_{g}\right)
\]
where
\[
\alpha_{g}=\left\{ -3,-3+h,-3+2h,\ldots,3-2h,3-h,3\right\} ,\,\,\, g=1,\ldots G,
\]
\[
s_{g}=\frac{h}{3}\left\{ 1,2,4,2,\ldots,2,4,1\right\} ,\,\,\, g=1,\ldots G,
\]
and
\[
w_{g}=f\left(\alpha_{g}\right).
\]
Notice that the {}``Simpson's weights'' $s_{g}$ are $2$ if the
element of the grid is even, and $4$ if odd, with $1$ as the first
and last element. \\
\\
\textbf{Notes:} This is quite computationally intensive, so if you
can't get estimates for the parameters, don't worry. I want to see
you code this. Of course, if you code it successfully, you should
estimate it; but it will take awhile, even on the cluster. Avoid loops
at all costs, because they go very slowly inside of Matlab's optimizers.
It takes a lot of thinking (and use of 3-dimensional matrices), but
this problem can be coded with no loops.\end{enumerate}
\end{enumerate}

\end{document}
